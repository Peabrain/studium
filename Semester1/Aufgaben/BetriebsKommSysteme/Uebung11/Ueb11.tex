\documentclass[a4paper]{scrartcl}
\usepackage[utf8]{inputenc}
\usepackage[ngerman]{babel}
\usepackage[T1]{fontenc}
\usepackage{mathtools}
\usepackage{amsmath}
\usepackage{amssymb}
\usepackage{amsfonts}
\usepackage{polynom}

\usepackage{tikz-timing}
\usetikztiminglibrary{clockarrows}

\title{Betriebs- und Kommunikationssysteme Übung 11}
\author{Andreas Timmermann, Mat: 4994606, Nhu Quynh, Mat: ..., Gruppe: 1}

\begin{document}
	\maketitle
	\begin{flushleft}
		\textbf{Aufgabe 1}\\
		\textbf{a)}\\
		\textbf{b)}\\

\begin{tikztimingtable}[scale=2,timing/.cd,
        c/dual arrows,c/arrow tip=latex,
        c/arrow pos=.7,
        metachar={v}{[timing/c/no arrows]c[timing/c/dual arrows]},
        slope=0]
    \shortstack[l]{Manchester-Codierung:\\(bi-phase)} 
      &h 0c0h0l 0c CCcvCcvCCc \\
 \extracode
   \begin{pgfonlayer}{background}
       \vertlines[help lines,brown]{}
       \foreach [count=\x] \b in {1,0,1,1,0,0,1,0} {
            \node [below,font=\sffamily\bfseries\tiny,inner ysep=2pt] at (\x-.5,0) {\b};
       }
   \end{pgfonlayer}
\end{tikztimingtable}		
	\end{flushleft}
\end{document}
