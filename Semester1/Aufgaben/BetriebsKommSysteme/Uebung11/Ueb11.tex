\documentclass[a4paper]{scrartcl}
\usepackage[utf8]{inputenc}
\usepackage[ngerman]{babel}
\usepackage[T1]{fontenc}
\usepackage{mathtools}
\usepackage{amsmath}
\usepackage{amssymb}
\usepackage{amsfonts}
\usepackage{polynom}

\usepackage{tikz-timing}

\title{Betriebs- und Kommunikationssysteme Übung 11}
\author{Andreas Timmermann, Mat: 4994606, Nhu Quynh, Mat: ..., Gruppe: 1}

\begin{document}
	\maketitle
	\begin{flushleft}
		\textbf{Aufgabe 1}\\
		\textbf{a)}\\
		\underline{Bit} – ist die kleinste Informationseinheit\\
		\underline{Signal} – ist die Hardwaremäßige Repräsentation einer Folge von Bits\\
		\underline{Frequenzmodulation} – hier werden die Bits in einem Signal durch Veränderung der Frequenz einer Sinuskurve dargestellt\\
		\underline{Amplitudenmodulation} – hier werden die Bits in einem Signal durch Veränderung der Amplitude einer Sinuskurve dargestellt\\
		\underline{Phasenmodulation} - hier werden die Bits in einem Signal durch Veränderung der Phase einer Sinuskurve dargestellt\\
		\textbf{b)}\\

\begin{tikzpicture}
    %red vertical lines
    \foreach \x in {0,1,...,15} {
        \draw[color=red] (\x,0.5) -- +(0,3);  };

    %black lines with arrows
    \draw[very thick] (0,2.5) -- +(0.5,0) -- +(0.5,-1.0)-- +(1.0,-1.0); %1
    \draw[very thick] (1,1.5) -- +(0.5,0.0)-- +(0.5,1.0)-- +(1.0,1.0)-- +(1.0,0.0); %0
    \draw[very thick] (2,1.5) -- +(0.5,0.0)-- +(0.5,1.0)-- +(1.0,1.0); %0
    \draw[very thick] (3,2.5) -- +(0.5,0) -- +(0.5,-1.0)-- +(1.0,-1.0); %1
    \draw[very thick] (4,1.5) -- +(0.5,0.0)-- +(0.5,1.0)-- +(1.0,1.0)-- +(1.0,0.0); %0
    \draw[very thick] (5,1.5) -- +(0.5,0.0)-- +(0.5,1.0)-- +(1.0,1.0)-- +(1.0,0.0); %0
    \draw[very thick] (6,1.5) -- +(0.5,0.0)-- +(0.5,1.0)-- +(1.0,1.0); %0
    \draw[very thick] (7,2.5) -- +(0.5,0) -- +(0.5,-1.0)-- +(1.0,-1.0); %1
    \draw[very thick] (8,1.5) -- +(0.5,0.0)-- +(0.5,1.0)-- +(1.0,1.0); %0
    \draw[very thick] (9,2.5) -- +(0.5,0) -- +(0.5,-1.0)-- +(1.0,-1.0)-- +(1.0,0.0); %1
    \draw[very thick] (10,2.5) -- +(0.5,0) -- +(0.5,-1.0)-- +(1.0,-1.0)-- +(1.0,0.0); %1
    \draw[very thick] (11,2.5) -- +(0.5,0) -- +(0.5,-1.0)-- +(1.0,-1.0)-- +(1.0,0.0); %1
    \draw[very thick] (12,2.5) -- +(0.5,0) -- +(0.5,-1.0)-- +(1.0,-1.0); %1
    \draw[very thick] (13,1.5) -- +(0.5,0.0)-- +(0.5,1.0)-- +(1.0,1.0); %0
    \draw[very thick] (14,2.5) -- +(0.5,0) -- +(0.5,-1.0)-- +(1.0,-1.0); %1

    %numbers
    \path (0.5,1) node{1};
    \path (1.5,1) node{0};
    \path (2.5,1) node{0};
    \path (3.5,1) node{1};
    \path (4.5,1) node{0};
    \path (5.5,1) node{0};
    \path (6.5,1) node{0};
    \path (7.5,1) node{1};
    \path (8.5,1) node{0};
    \path (9.5,1) node{1};
    \path (10.5,1) node{1};
    \path (11.5,1) node{1};
    \path (12.5,1) node{1};
    \path (13.5,1) node{0};
    \path (14.5,1) node{1};
\end{tikzpicture}
	Darstellung wie in 9.23 der Vorlesung.\\

	\end{flushleft}
\end{document}
