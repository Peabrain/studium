\documentclass[a4paper]{scrartcl}
\usepackage[utf8]{inputenc}
\usepackage[ngerman]{babel}
\usepackage[T1]{fontenc}
\usepackage{mathtools}
\usepackage{amsmath}
\usepackage{amssymb}
\usepackage{amsfonts}
\usepackage{polynom}

\title{Logik und diskrete Mathematik, Übung 10}
\author{Andreas Timmermann, Mat: 4994606, Alena Dudarenok, Mat: 4999780, Gruppe: 3}

\begin{document}
	\maketitle
	\begin{flushleft}
		\textbf{Aufgabe 1}\\

		\textbf{a)}\\
		geg: ein Bitstring mit 10 Nullen ($0$) und 12 Einsen ($1$).\\
		ges: Wieviele Kombinationen $K$ gibt es, wenn hinter jeder $0$ eine $1$ stehen muss?\\[1em]
		Der Basisstring ist $1101010101010101010101$. Die ersten beiden $1$ werden jeweils zwischen $01$ geschoben. Das garantiert die Ausgangsbedingung. Also \\[1em]
		$1101010101010101010101$\\
		$1011010101010101010101$\\
		$1010110101010101010101$\\
		$1010101101010101010101$\\
		$1010101011010101010101$\\
		$1010101010110101010101$\\
		$1010101010101101010101$\\
		$1010101010101011010101$\\
		$1010101010101010110101$\\
		$1010101010101010101101$\\
		$1010101010101010101011$\\[1em]
		Dann verschiebe ich beidere vorderen Bits um eine $0$ weiter und das Spiel beginnt von vorne.\\[1em]
		$0111010101010101010101$\\
		$0110110101010101010101$\\
		$0110101101010101010101$\\
		$0110101011010101010101$\\
		$0110101010110101010101$\\
		$0110101010101101010101$\\
		$0110101010101011010101$\\
		$0110101010101010110101$\\
		$0110101010101010101101$\\
		$0110101010101010101011$\\[1em]
		Und so weiter ... Damit ist die Kombinationsmöglichkeit $K$ bestimmt.\\
		$K=11+10+9+8+7+6+5+4+3+2+1$\\[1em]

		\textbf{b)}\\
		geg: $n$ unabhängige Bernoulli Versuche.\\
		Wir benutzen $b(k,n,p)=\binom{n}{k}\cdot p^k\cdot (1-p)^{n-k}$\\[1em]
		\textbf{(1.)} Wahrscheinlichkeit $p_a$ für keinen einzigen Misserfolg (ME).\\[1em]
		$k=n \Rightarrow p_a=\binom{n}{n}\cdot p^n\cdot (1-p)^0=p^n$\\[1em]
		\textbf{(2.)} Wahrscheinlichkeit $p_b$ für mindestens einen Misserfolg (ME).\\[1em]
		Da $p_a$ die Wahrscheinlichkeit für alle Erfolge ist, ist $p_b=1-p_a$ die Wahrscheinlichkeit mindestens 1 Misserfolg zu haben.\\[1em]		
		\textbf{(3.)} Wahrscheinlichkeit $p_c$ für höchstens 2 Misserfolg (ME).\\[1em]
		$p_c=b(n,n,p)+b(n-1,n,p)+b(n-2,n,p)$\\
		$= p_a+\binom{n}{n-1}p^{n-1}\cdot (1-p)^{n-(n-1)}+\binom{n}{n-2}p^{n-2}\cdot (1-p)^{n-(n-2)}$\\
		$=p_a+n\cdot p^{n-1}\cdot (1-p)+\binom{n}{n-2}\cdot p^{n-2}\cdot (1-p)^2$\\
		\textbf{(4.)} Wahrscheinlichkeit $p_d$ für mindestens 2 Misserfolg (ME).\\[1em]
		$p_d=1-(p_a+n\cdot p^{n-1}\cdot (1-p))$\\[1em]

		\textbf{c)}\\
		zu zeigen: $\binom{2\cdot n}{n} + \binom{2\cdot n}{n - 1} = \frac{1}{2}\cdot\binom{2\cdot n + 2}{n + 1}$\\[1em]
		Als erstes multipliziere ich beide Seiten mit $2$, welches ich am Ende wieder rückgängig mache.\\
		$2\cdot(\binom{2\cdot n}{n} + \binom{2\cdot n}{n - 1}) = \binom{2\cdot n + 2}{n + 1}$\\[1em]
		$2\cdot(\binom{2\cdot n}{n} + \binom{2\cdot n}{n - 1})$\\
		$= 2\cdot\binom{2\cdot n}{n} + 2\cdot\binom{2\cdot n}{n - 1}$\\
		$= \binom{2\cdot n}{n} + \binom{2\cdot n}{n} + \binom{2\cdot n}{n - 1} + \binom{2\cdot n}{n - 1}$\\
		$= \binom{2\cdot n}{n} + \binom{2\cdot n}{n - 1} + \binom{2\cdot n}{n} + \binom{2\cdot n}{n - 1}$\\
		$= \binom{2\cdot n}{n} + \binom{2\cdot n}{n - 1} + \binom{2\cdot n}{n} + \binom{2\cdot n}{n + 1}$\\
		$= \binom{2\cdot n + 1}{n} + \binom{2\cdot n + 1}{n + 1}$\\
		$= \binom{2\cdot n + 2}{n + 1}$\\[1em]
		Und jetzt wird wieder durch 2 dividiert und das ist $= \frac{1}{2}\binom{2\cdot n + 2}{n + 1}$\\[1em]

		\textbf{d)}\\
		zu zeigen: $Pr(E\cap F) \geq Pr(E)+Pr(F)-1$.\\
		geg: $0\leq Pr(E),Pr(F),Pr(E\cap F)\leq 1$ und $Pr(E\cap F)=Pr(E)\cdot Pr(F)$\\[1em]
		Zuerst legen wir alles auf eine Seite, um etwas Übersicht zu bekommen.\\
		$1 - Pr(E) - Pr(F) + Pr(E)\cdot Pr(F) \geq 0$\\
		Jetzt noch etwas zusammenfassen und ...\\
		$1 - 1\cdot(Pr(E) + Pr(F)) + Pr(E)\cdot Pr(F) \geq 0$\\
		es ist schon eine Struktur erkennbar.\\
		$1^2 - 1\cdot(Pr(E) + Pr(F)) + Pr(E)\cdot Pr(F) =$\\
		$= (1-Pr(E))\cdot(1-Pr(F))\geq 0$\\[1em]
		Und das ist mit der Voraussetzung gegeben und somit ist dir Formel erfüllt.\\[1em]
	\end{flushleft}
	\begin{flushleft}
		\textbf{Aufgabe 2}\\
		\textbf{geg:} Wahrscheinlichkeitsraum $(\Omega,p)$ mit $(i,j)\in \Omega=\{1,2,3,4,5,6\}\times\{1,2,3,4,5,6\}$ und $p=\frac{1}{36}$.\\
		Zufallsvariablen:\\
		$X(i,j)=|i-j|$\\
		$Y(i,j)=5\cdot i+j$\\
		$Z(i,j)=5\cdot max(i,j)+min(i,j)$\\[1em]
		ges: Beschreibung der Verteilungsfunktionen $X,Y,Z$ und Ermittlung der Erwartungswerte $E(X),E(Y),E(Z)$.\\[1em]
		
		\begin{center}
		Verteilungsfunktion $X(i,j)$\\
		\begin{tabular}{|r|r|r|r|r|r|r|}
		\hline
		\multicolumn{1}{|c|}{$j\backslash i$} & \multicolumn{1}{c|}{\textbf{1}} & \multicolumn{1}{c|}{\textbf{2}} & \multicolumn{1}{c|}{\textbf{3}} & \multicolumn{1}{c|}{\textbf{4}} & \multicolumn{1}{c|}{\textbf{5}} & \multicolumn{1}{c|}{\textbf{6}} \\ \hline
		\textbf{1} & 0 & 1 & 2 & 3 & 4 & 5 \\ \hline
		\textbf{2} & 1 & 0 & 1 & 2 & 3 & 4 \\ \hline
		\textbf{3} & 2 & 1 & 0 & 1 & 2 & 3 \\ \hline
		\textbf{4} & 3 & 2 & 1 & 0 & 1 & 2 \\ \hline
		\textbf{5} & 4 & 3 & 2 & 1 & 0 & 1 \\ \hline
		\textbf{6} & 5 & 4 & 3 & 2 & 1 & 0 \\ \hline
		\end{tabular}
		\end{center}
		$Im(X)=\{0,1,2,3,4,5\}$\\
		$E(X)=\sum\limits_{\omega\in\Omega} X(\omega)\cdot Pr(\omega) = 1.9\overline{4}$\\[1em]

		\begin{center}
		Verteilungsfunktion $Y(i,j)$\\
		\begin{tabular}{|r|r|r|r|r|r|r|}
		\hline
		\multicolumn{1}{|c|}{$j\backslash i$} & \multicolumn{1}{c|}{\textbf{1}} & \multicolumn{1}{c|}{\textbf{2}} & \multicolumn{1}{c|}{\textbf{3}} & \multicolumn{1}{c|}{\textbf{4}} & \multicolumn{1}{c|}{\textbf{5}} & \multicolumn{1}{c|}{\textbf{6}} \\ \hline
		\textbf{1} & 6  & 11 & 16 & 21 & 26 & 31 \\ \hline
		\textbf{2} & 7  & 12 & 17 & 22 & 27 & 32 \\ \hline
		\textbf{3} & 8  & 13 & 18 & 23 & 28 & 33 \\ \hline
		\textbf{4} & 9  & 14 & 19 & 24 & 29 & 34 \\ \hline
		\textbf{5} & 10 & 15 & 20 & 25 & 30 & 35 \\ \hline
		\textbf{6} & 11 & 16 & 21 & 26 & 31 & 36 \\ \hline
		\end{tabular}
		\end{center}
		$Im(Y)=\{6,7,\cdots,36\}$\\
		$E(Y)=\sum\limits_{\omega\in\Omega} X(\omega)\cdot Pr(\omega) = 21$\\[1em]

		\begin{center}
		Verteilungsfunktion $Z(i,j)$\\
		\begin{tabular}{|r|r|r|r|r|r|r|}
		\hline
		\multicolumn{1}{|c|}{$j\backslash i$} & \multicolumn{1}{c|}{\textbf{1}} & \multicolumn{1}{c|}{\textbf{2}} & \multicolumn{1}{c|}{\textbf{3}} & \multicolumn{1}{c|}{\textbf{4}} & \multicolumn{1}{c|}{\textbf{5}} & \multicolumn{1}{c|}{\textbf{6}} \\ \hline
		\textbf{1} & 6   & 11 & 16 & 21 & 26 & 31 \\ \hline
		\textbf{2} & 11  & 12 & 17 & 22 & 27 & 32 \\ \hline
		\textbf{3} & 16  & 17 & 18 & 23 & 28 & 33 \\ \hline
		\textbf{4} & 21  & 22 & 23 & 24 & 29 & 34 \\ \hline
		\textbf{5} & 26  & 27 & 28 & 29 & 30 & 35 \\ \hline
		\textbf{6} & 31  & 32 & 33 & 34 & 35 & 36 \\ \hline
		\end{tabular}
		\end{center}
		$Im(Z)=\{6,11,12,16,17,18,21,22,23,24,26,27,28,29,30,31,32,33,34,35,36\}$\\
		$E(Z)=\sum\limits_{\omega\in\Omega} X(\omega)\cdot Pr(\omega) = 24.\overline{8}$\\[1em]
		
	\end{flushleft}
	\begin{flushleft}
		\textbf{Aufgabe 3}\\
		geg: $Pr(A)=\frac{3}{5}, Pr(B)=\frac{1}{3}$\\
		Riehenfolge: $BAABBAABBAABBAABBA....$\\
		ges: Wahrscheinlichkeit $Pr$, dass $Bob(B)$ also erstes Trifft.\\[1em]
		$Pr=\frac{1}{3}+(\frac{2}{3})^1\cdot(\frac{2}{5})^2\cdot\frac{1}{3}+(\frac{2}{3})^2\cdot(\frac{2}{5})^2\cdot\frac{1}{3}+(\frac{2}{3})^3\cdot(\frac{2}{5})^4\cdot\frac{1}{3}+(\frac{2}{3})^4\cdot(\frac{2}{5})^4\cdot\frac{1}{3}+\cdots=$\\
		$=\frac{1}{3}+\sum\limits_{i=0}^{\infty}((\frac{2}{3})^{2\cdot i+1}\cdot(\frac{2}{5})^{2\cdot i+2}\cdot\frac{1}{3} + (\frac{2}{3})^{2\cdot i+2}\cdot(\frac{2}{5})^{2\cdot i+2}\cdot\frac{1}{3})$\\
		$=\frac{1}{3}+\frac{1}{3}\cdot\sum\limits_{i=0}^{\infty}((\frac{2}{3})^{2\cdot i+1}\cdot(\frac{2}{5})^{2\cdot i+2} + (\frac{2}{3})^{2\cdot i+2}\cdot(\frac{2}{5})^{2\cdot i+2})$\\
		$=\frac{1}{3}+\frac{1}{3}\cdot\sum\limits_{i=0}^{\infty}(\frac{2}{5})^{2\cdot i+2}\cdot((\frac{2}{3})^{2\cdot i+1} + (\frac{2}{3})^{2\cdot i+2})$\\
		$=\frac{1}{3}+\frac{1}{3}\cdot\sum\limits_{i=0}^{\infty}(\frac{2}{5})^{2\cdot i+2}\cdot((\frac{2}{3})^{2\cdot i+1} + (\frac{2}{3})\cdot(\frac{2}{3})^{2\cdot i+1})$\\
		$=\frac{1}{3}+\frac{1}{3}\cdot\sum\limits_{i=0}^{\infty}(\frac{2}{5})^{2\cdot i+2}\cdot(\frac{2}{3})^{2\cdot i+1}\cdot(1 + \frac{2}{3})$\\
		$=\frac{1}{3}+\frac{1}{3}\cdot\sum\limits_{i=0}^{\infty}(\frac{2}{5})^{2\cdot i+2}\cdot(\frac{2}{3})^{2\cdot i+1}\cdot\frac{5}{3}$\\
		$=\frac{1}{3}+\frac{5}{9}\cdot\sum\limits_{i=0}^{\infty}(\frac{2}{5})^{2\cdot i+2}\cdot(\frac{2}{3})^{2\cdot i+1}$\\
		$=\frac{1}{3}+\frac{5}{9}\cdot\sum\limits_{i=0}^{\infty}\frac{2}{5}\cdot(\frac{2}{5})^{2\cdot i+1}\cdot(\frac{2}{3})^{2\cdot i+1}$\\
		$=\frac{1}{3}+\frac{10}{45}\cdot\sum\limits_{i=0}^{\infty}(\frac{4}{15})^{2\cdot i+1}$\\ 
		$=\frac{1}{3}+\frac{10}{45}\cdot \frac{4}{15}\cdot\sum\limits_{i=0}^{\infty}(\frac{4}{15})^{2\cdot i}$\\ 	
		$=\frac{1}{3}+\frac{10}{45}\cdot \frac{4}{15}\cdot\frac{1}{1-(\frac{4}{15})^2}$\\ 	
		$=\frac{83}{209}$\\
		\end{flushleft}
\end{document}
