\documentclass[a4paper]{scrartcl}
\usepackage[utf8]{inputenc}
\usepackage[ngerman]{babel}
\usepackage[T1]{fontenc}
\usepackage{mathtools}
\usepackage{amsmath}
\usepackage{amssymb}
\usepackage{amsfonts}
\usepackage{polynom}

\title{Logik und diskrete Mathematik, Übung 5}
\author{Andreas Timmermann, Mat: 4994606, Alena Dudarenok, Mat: 4999780, Gruppe: 1}

\begin{document}
	\maketitle
	\begin{flushleft}
		\textbf{Aufgabe 1:}\\
		geg:\\
		$f:A\rightarrow B$ ist eine Abbildung. $\forall S,T\subseteq A$ gilt $f(S\cap T)=f(S)\cap f(T)$.\\
		zu zeigen: $f$ ist injektiv.\\[1em]
		$M_1 = S_1\cap T_1,M_2 = S_2\cap T_2$\\
		$f$ ist injektiv, wenn $\forall M_1,M_2$ gilt $f(M_1) = f(M_2)\Rightarrow M_1 = M_2$.\\ 
		$\equiv f(S_1)\cap f(T_1) = f(S_2)\cap f(T_2)\Rightarrow M_1 = M_2$\\[1em]
		Kontraposition:\\
		$M_1 \neq M_2 \Rightarrow f(M_1) \neq f(M_2)$\\
		o.B.d.A:\\
		$M_1 \neq M_2 \Leftrightarrow \exists x\in M_1: x\nexists M_2$\\
		$\Rightarrow (x\in S_1 \wedge x\in T_1) \wedge (x\notin S_2 \vee x\notin T_2)$\\
		Das bedeutet, dass an der Stelle $x$ nur ein $f(x)$ für $f(M_1)$ existiert, aber nicht für $f(M_2)$. Und da es für jedes beliebige $x$ gilt, für das unsere "o.B.d.A" gilt ...\\
		$\Rightarrow f(M_1)\neq f(M_2)$\\
		Somit ist die Kontraposition wahr und die Aussage auch. $f$ ist injektiv.
	\end{flushleft}
	\begin{flushleft}
		\textbf{Aufgabe 2:}\\
		geg: $f:A\rightarrow B$ und $g: \mathbb{P}(A)\rightarrow \mathbb{P}(A)$. Und für $N\subseteq B$ gilt $g(N) = f^{-1}(N)$.\\
		zu zeigen: Sei $f$ surjektiv $\Leftrightarrow$ $g$ injektiv.\\[1em]
		$"\Leftarrow"$:\\
		Wir beweisen hier durch die Kontraposition:\\
		$f$ nicht surjektiv $\Rightarrow$ $g$ nicht injektiv.\\
		Wenn $f$ nicht surjektiv $\Rightarrow$ $\exists y\in B: y \nexists f(A)$.\\
		Dann ist aber $g(B) = g(B\backslash\{x\})=A \wedge B\neq B\backslash \{x\}$\\
		Und das bedeutet, dass $g$ nicht injektiv sein kann.\\[1em]
		$"\Rightarrow"$:\\
		wieder Beweis durch Kontraposition.\\
		$g$ nicht injektiv $\Rightarrow$ $f$ nicht surjektiv.\\
		Wenn $g$ nicht surjektiv ist $\Rightarrow$ $\exists N_1,N_2 \subseteq B$ mit $N_1\neq N2 \wedge g(N_1)=g(N_2)$\\
		Nach der Definition von $g$ gilt dann auch $f^{-1}(N_1)=f^{-1}(N_2)$.\\
		Da $N_1\neq N_2$, nehmen wir an (o.D.b.A), $\exists x_2\in N_2\backslash N_1$.\\
		Dann gilt aber $x_2\neq f(A)$, weil wenn $x_2$ ein nichtleeres Urbild mit einem $a_2$ hätte, dann wäre auch $a_2$ im Urbild eines $x_1\in N_1$. Da aber $f(a)$ nur aus einem Element besteht, kann $f$ nicht surjektiv sein.\\[1em]
	\end{flushleft}
	\begin{flushleft}
		\textbf{Aufgabe 3:}\\		
		(a) Nein, wegen Funktionen mit Grenzwert.\\
		(b) Ja, weil stetig = es gibt keine Unterbrechungen und bijektiv = zu jedem $f(x)$ wird genau 1 eindeutiges  $x$ zugewiesen und umgedreht genauso.\\
		(c) Wenn man die Funktion $y = \tan(x), \forall x\in \mathbb{R}$ durch das Intervall auf x = $[-\frac{pi}{2},\frac{pi}{2})$ beschränkt, dann ist $M$ beschränkt, aber $f(M)=\{-\infty,\infty\}$ ist nicht beschränkt.\\
		(d) Nein. Siehe Beispiel (c).\\  
		(e) Nein. Siehe Beispiel (c).\\  
		(f) Nein, da  $f(x)=\frac{1}{x}$ ist injektiv aber nicht streng monoton. 
	\end{flushleft}
	\begin{flushleft}
		\textbf{Aufgabe 4:}\\
		a)\\
		zu zeigen: $B^n, B=\{0,1\}, n\in \mathbb{N}$ ist abzählbar unendlich.\\[1em]
		$B^1 = B = \{0,1\} mit  |B| = 2^1, B^2 = B\times B = \{0,1\}\times \{0,1\} mit  |B^2| = 2^2$\\ sind abzählbar.\\
		Also $|B^n| = 2^n$. Und jede Vereinigung abzählbarer Menge ist abzählbar. Und für beliebige $k$'s auch unendlich abzählbar.\\ [1em]
		b)\\
		Behauptung:  $|B^{\mathbb{N}}| = |\mathbb{P}(\mathbb{N})|$\\
		Anwendung des Cantor Diagonalisierungsverfahrens:\\ 
		$B_0 = \underline{0}100101\dots$\\
		$B_1 = 1\underline{0}01100\dots$\\
		$B_2 = 11\underline{0}1101\dots$\\
		$B_3 = 011\underline{0}100\dots$\\
		$B_4 = 0000\underline{1}11\dots$\\
		$B_5 = 11000\underline{0}1\dots$\\
		$B_6 = 011110\underline{1}\dots$\\
		$\vdots$\\
		Bildung der Diagonalen: $b = 0000101\dots$. Wir definieren die Funktion $f(b) = \{b_i = \left\{ \begin{array}{rcl}
		         0 & \mbox{:} & b_i = 1 \\
		         1 & \mbox{:} & b_i = 0 \\
                \end{array}\right.  : i\in \mathbb{N}\}$\\
        $b' = 1111010\dots$ ist verschieden von allen $B_k$'s und somit ist eine neue 0-1-Sequenz generiert worden, die vorher noch nicht abzählbar war.\\
        Hier haben wir, wie im Beispiel von $\mathbb{R}$, auch die Überabzählbarkeit  von $B^{\mathbb{N}}$gezeigt.\\[1em]
        Andere Möglichkeit wäre die 0-1-Sequenzen durch eine bijektive Funktion in reelle Zahlen umzuwandeln, um dann auch mit dem Cantor Diagonalisierungsverfahren die Überabzählbarkeit nachzuweisen.\\
        Da die Funktion bijektiv ist, ist die Beweisführung gleich.\\
	\end{flushleft}
	\begin{flushleft}
		\textbf{Aufgabe 5:}\\		
		zu zeigen: $\forall a,b\in \mathbb{Z}: \neg(3|a) \vee \neg(3|b) \Rightarrow \neg(3|(a+b)) \vee \neg(3|(a-b))$\\ 
		Kontraposition:\\
		$\forall a,b\in \mathbb{Z}: 3|(a+b) \wedge 3|(a-b) \Rightarrow 3|a \wedge 3|b$\\[1em]
		Voraussetzung:\\
		$a = 3\times m_a + r_a$. Nach Definition $3|a$, genau dann wenn $r_a = 0$.\\ 
		$b = 3\times m_b + r_b$. Nach Definition $3|b$, genau dann wenn $r_b = 0$.\\[1em]
		Wir setzen nun ein ohne zu wissen, wie groß $r_a,r_b$ sind.\\
		$3|(3\times m_a + r_a + 3\times m_b + r_b) \wedge 3|(3\times m_a + r_a - 3\times m_b - r_b)$\\
		$\equiv 3|(3\times (m_a + m_b) + (r_a + r_b)) \wedge 3|(3\times (m_a - m_b) + (r_a - r_b))$\\
		Da wir wissen, dass $3|(3\times (m_a + m_b))$ und $3|(3\times (m_a - m_b))$ wahr ist, schauen wir uns $r_a,r_b$ an.\\
		Um die Voraussetzung zu erfüllen, muss folgendes gelten: $(r_a + r_b) = (r_a - r_b) = 0$\\
		$\equiv (r_a + r_b) = r_a = r_b$. Und das erfüllt nur die das neutrale Element der Addition der ganzen Zahlen, die $0$.\\
		Somit gilt $(r_a + r_b) = r_a = r_b = 0$.\\
		$\Rightarrow 3|a = 3|(3\times m_a + 0), 3|b = 3|(3\times m_b + 0)$ ist wahr und die Kontraposition und die zugrundeliegende Aussage auch.
		
	\end{flushleft}
\end{document}
