\documentclass[a4paper]{scrartcl}
\usepackage[utf8]{inputenc}
\usepackage[ngerman]{babel}
\usepackage[T1]{fontenc}
\usepackage{mathtools}
\usepackage{amsmath}
\usepackage{amssymb}
\usepackage{amsfonts}
\usepackage{polynom}

\title{Logik und diskrete Mathematik, Übung 5}
\author{Andreas Timmermann, Mat: 4994606, Alena Dudarenok, Mat: 4999780, Gruppe: 1}

\begin{document}
	\maketitle
	\begin{flushleft}
		\textbf{Aufgabe 1:}\\
		geg:\\
		$f:A\rightarrow B$ ist eine Abbildung. $\forall S,T\subseteq A$ gilt $f(S\cap T)=f(S)\cap f(T)$.\\
		zu zeigen: $f$ ist injektiv.\\[1em]
		$M_1 = S_1\cap T_1,M_2 = S_2\cap T_2$\\
		$f$ ist injektiv, wenn $\forall M_1,M_2$ gilt $f(M_1) = f(M_2)\Rightarrow M_1 = M_2$.\\ 
		$\equiv f(S_1)\cap f(T_1) = f(S_2)\cap f(T_2)\Rightarrow M_1 = M_2$\\[1em]
		Kontraposition:\\
		$M_1 \neq M_2 \Rightarrow f(M_1) \neq f(M_2)$\\
		o.B.d.A:\\
		$M_1 \neq M_2 \Leftrightarrow \exists x\in M_1: x\nexists M_2$\\
		$\Rightarrow (x\in S_1 \wedge x\in T_1) \wedge (x\notin S_2 \vee x\notin T_2)$\\
		Das bedeutet, dass an der Stelle $x$ nur ein $f(x)$ für $f(M_1)$ existiert, aber nicht für $f(M_2)$. Und da es für jedes beliebige $x$ gilt, für das unsere "o.B.d.A" gilt ...\\
		$\Rightarrow f(M_1)\neq f(M_2)$\\
		Somit ist die Kontraposition wahr und die Aussage auch. $f$ ist injektiv.
	\end{flushleft}
	\begin{flushleft}
		\textbf{Aufgabe 2:}\\
	\end{flushleft}
	\begin{flushleft}
		\textbf{Aufgabe 3:}\\		
		(a)
		(b) Ja, weil stetig = es gibt keine Unterbrechungen und bijektiv = zu jedem $f(x)$ wird genau 1 eindeutiges  $x$ zugewiesen und umgedreht genauso.\\
		(c) Wenn man die Funktion $y = \tan(x), \forall x\in \mathbb{R}$ durch das Intervall auf x = $[-\frac{pi}{2},\frac{pi}{2})$ beschränkt, dann ist $M$ beschränkt, aber $f(M)=\{-\infty,\infty\}$ ist nicht beschränkt.\\
		(d) Nein. Siehe Beispiel (d).\\  
		(e) Nein. Siehe Beispiel (c).\\  
		(f) Ja, da  durch die Injektivität schon gegeben ist, zu jedem $f(x)$ nur ein $x$ geben kann. 
	\end{flushleft}
	\begin{flushleft}
		\textbf{Aufgabe 4:}\\
		zu zeigen: $B^n, B=\{0,1\}, n\in \mathbb{N}$ ist abzählbar unendlich.\\[1em]
		$B^1 = B = \{0,1\} mit  |B| = 2^1, B^2 = B\times B = \{0,1\}\times \{0,1\} mit  |B^2| = 2^2$\\ sind abzählbar.\\
		Also $|B^n| = 2^n$. Und jede Vereinigung abzählbarer Menge ist abzählbar. Und für beliebige $k$'s auch unendlich abzählbar.\\		
	\end{flushleft}
	\begin{flushleft}
		\textbf{Aufgabe 5:}\\		
		zu zeigen: $\forall a,b\in \mathbb{Z}: \neg(3|a) \vee \neg(3|b) \Rightarrow \neg(3|(a+b)) \vee \neg(3|(a-b))$\\ 
		Kontraposition:\\
		$\forall a,b\in \mathbb{Z}: 3|(a+b) \wedge 3|(a-b) \Rightarrow 3|a \wedge 3|b$\\[1em]
		Voraussetzung:\\
		$a = 3\times m_a + r_a$. Nach Definition $3|a$, genau dann wenn $r_a = 0$.\\ 
		$b = 3\times m_b + r_b$. Nach Definition $3|b$, genau dann wenn $r_b = 0$.\\[1em]
		Wir setzen nun ein ohne zu wissen, wie groß $r_a,r_b$ sind.\\
		$3|(3\times m_a + r_a + 3\times m_b + r_b) \wedge 3|(3\times m_a + r_a - 3\times m_b - r_b)$\\
		$\equiv 3|(3\times (m_a + m_b) + (r_a + r_b)) \wedge 3|(3\times (m_a - m_b) + (r_a - r_b))$\\
		Da wir wissen, dass $3|(3\times (m_a + m_b))$ und $3|(3\times (m_a - m_b))$ wahr ist, schauen wir uns $r_a,r_b$ an.\\
		Um die Voraussetzung zu erfüllen, muss folgendes gelten: $(r_a + r_b) = (r_a - r_b) = 0$\\
		$\equiv (r_a + r_b) = r_a = r_b$. Und das erfüllt nur die das neutrale Element der Addition der ganzen Zahlen, die $0$.\\
		Somit gilt $(r_a + r_b) = r_a = r_b = 0$.\\
		$\Rightarrow 3|a = 3|(3\times m_a + 0), 3|b = 3|(3\times m_b + 0)$ ist wahr und die Kontraposition und die zugrundeliegende Aussage auch.
		
	\end{flushleft}
\end{document}
