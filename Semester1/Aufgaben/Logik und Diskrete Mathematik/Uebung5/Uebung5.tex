\documentclass[a4paper]{scrartcl}
\usepackage[utf8]{inputenc}
\usepackage[ngerman]{babel}
\usepackage[T1]{fontenc}
\usepackage{mathtools}
\usepackage{amsmath}
\usepackage{amssymb}
\usepackage{amsfonts}
\usepackage{polynom}

\title{Logik und diskrete Mathematik, Übung 5}
\author{Andreas Timmermann, Mat: 4994606, Alena Dudarenok, Mat: 4999780, Gruppe: 1}

\begin{document}
	\maketitle
	\begin{flushleft}
		\textbf{Aufgabe 1:}\\
	\end{flushleft}
	\begin{flushleft}
		\textbf{Aufgabe 2:}\\
	\end{flushleft}
	\begin{flushleft}
		\textbf{Aufgabe 3:}\\		
	\end{flushleft}
	\begin{flushleft}
		\textbf{Aufgabe 4:}\\		
	\end{flushleft}
	\begin{flushleft}
		\textbf{Aufgabe 5:}\\		
		zu zeigen: $\forall a,b\in \mathbb{Z}: \neg(3|a) \vee \neg(3|b) \Rightarrow \neg(3|(a+b)) \vee \neg(3|(a-b))$\\ 
		Kontraposition:\\
		$\forall a,b\in \mathbb{Z}: 3|(a+b) \wedge 3|(a-b) \Rightarrow 3|a \wedge 3|b$\\[1em]
		Voraussetzung:\\
		$a = 3\times m_a + r_a$. Nach Definition $3|a$, genau dann wenn $r_a = 0$.\\ 
		$b = 3\times m_b + r_b$. Nach Definition $3|b$, genau dann wenn $r_b = 0$.\\[1em]
		Wir setzen nun ein ohne zu wissen, wie groß $r_a,r_b$ sind.\\
		$3|(3\times m_a + r_a + 3\times m_b + r_b) \wedge 3|(3\times m_a + r_a - 3\times m_b - r_b)$\\
		$\equiv 3|(3\times (m_a + m_b) + (r_a + r_b)) \wedge 3|(3\times (m_a - m_b) + (r_a - r_b))$\\
		Da wir wissen, dass $3|(3\times (m_a + m_b))$ und $3|(3\times (m_a - m_b))$ wahr ist, schauen wir uns $r_a,r_b$ an.\\
		Um die Voraussetzung zu erfüllen, muss folgendes gelten: $(r_a + r_b) = (r_a - r_b) = 0$\\
		$\equiv (r_a + r_b) = r_a = r_b$. Und das erfüllt nur die das neutrale Element der Addition der ganzen Zahlen, die $0$.\\
		Somit gilt $(r_a + r_b) = r_a = r_b = 0$.\\
		$\Rightarrow 3|a = 3|(3\times m_a + 0), 3|b = 3|(3\times m_b + 0)$ ist wahr und die Kontraposition und die zugrundeliegende Aussage auch.
		
	\end{flushleft}
\end{document}
