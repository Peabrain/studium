\documentclass[a4paper]{scrartcl}
\usepackage[utf8]{inputenc}
\usepackage[ngerman]{babel}
\usepackage[T1]{fontenc}
\usepackage{mathtools}
\usepackage{amsmath}
\usepackage{amssymb}
\usepackage{amsfonts}
\usepackage{polynom}

\title{Logik und diskrete Mathematik, Übung 6}
\author{Andreas Timmermann, Mat: 4994606, Alena Dudarenok, Mat: 4999780, Gruppe: 1}

\begin{document}
	\maketitle
	\begin{flushleft}
		\textbf{Aufgabe 1a:}\\
		geg: sind 6 Computer mit jeweil 0 bis 5 Verbindungen zu jeweils anderen Computer. Die Verbindungen sind bidirektional.\\
		zu zeigen: Es gibt mindestens 2 Computer mit der gleichen Anzahl von Verbindungen.\\[1em]
		\textbf{Widerspruchsbeweis:}\\[1em]
		Nehmen an, dass es keine 2 Computer gibt, die die gleiche Verbindungsanzahl haben und führen das zum Widerspruch.\\[1em]
		Eine veranschaulichung für eine beispielhafte Verbingung der Computer a bis b untereinander.\\
		\begin{tabular}{|c|c|c|c|c|c|}
			a & b & c & d & e & f \\
			(a,b) & (b,a) & (c,a) & (d,a) & (e,a) &   \\
			(a,c) & (b,c) & (c,b) & (d,b) &   &   \\
			(a,d) & (b,d) &       &       &   &   \\
			(a,e) &       &       &       &   &   \\
		 \end{tabular}\\[1em]
		 
		 Wir versuchen ein extremes Beispiel zu konstruieren, indem wir jedem Computer eine andere Anzahl von Verbindungen geben.\\
		 \textbf{Fall 1:}\\
		 Verbindungen der Computer zu anderen:\\
		 |a| = 4,|b| = 3,|c| = 2,|d| = 1,|e| = [0..4],|f| = 0 $\Rightarrow$ e muss [0..4] Verbindungen haben, weil 5 geht nicht, da ein Computer keine Verbindungen hat und der Computer zu sich selber keine Verbindung aufbauen kann.\\
		 \textbf{Fall 2:}\\
		 Verbindungen der Computer zu anderen:\\
		 |a| = 5,|b| = 4,|c| = 3,|d| = 2,|e| = [1..5],|f| = 1 $\Rightarrow$ e muss[1..5] Verbindungen haben, da es keinen Computer mit 0 Verbindungen gibt und somit jeder mindestens 1 Verbindung haben muss.\\
		 \textbf{Fall 3:}\\
		 Alle Computer haben weniger als 4 Verbindungen. $\Rightarrow$ Trivial. Hier muss es mindestens 2 Computer mit der gleichen Anzahl an Verbindungen geben.\\[1em]
		 Da die Widerspruchsannahme zum Widerspruch geführt wurde, muss die angenommene Aussage wahr sein.\\[1em]
	\end{flushleft}
	\begin{flushleft}
		\textbf{Aufgabe 1b:}\\
		geg: 
		seien $n_1,n_2,\cdots,n_t\in\mathbb{N}^+$\\
		sei $N = n_1+n_2+\cdots+n_t-t+1$ die Anzahl der Süßigkeiten.\\
		$t$ ist dabei die Anzahl der Stiefel.\\
		$S_1,S_2,\cdots,S_t$ sind die Stiefel.\\ [1em]
		zu zeigen:\\
		mindestens 1 $S_i$ hat mindestens $n_i$ Süßigkeiten.\\
		\textbf{Widerspruchsbeweis}\\
		$N = n_1+n_2+\cdots+n_t-t+1 = (n_1-1)+(n_2-1)+\cdots+(n_t-1)+1$\\
		Um den Widerspruch zu erfüllen, darf in jedem $S_i$ maximal $n_i-1$ Süßigkeiten drin sein. Da aber $N$ $+1$ Süßigkeiten mehr hat, muss 1 $S_i$ eine Süßgkeit mehr bekommen. Somit ist der Widerspruch falsch und die Aussage wahr.\\[1em]
	\end{flushleft}
	\begin{flushleft}
		\textbf{Aufgabe 2:}\\
		geg:\\
		$G = 20\cdot a + 50\cdot b$\\
		ges:\\
		Die Menge der $G$'s.\\
		\textbf{Behauptung:}\\
		Menge der auszahlbaren Geldbeträge ist: $M_G=\{n\mathbb{N}:10|n,n\geq 20,n \neq 30\}$\\[1em]
		Induktionsanfang:\\
		$a=1,b=0, G=20\cdot 1 + 50\cdot 0 \in M_G$\\
		$a=0,b=1, G=20\cdot 0 + 50\cdot 1 \in M_G$\\
		$a=1,b=1, G=20\cdot 1 + 50\cdot 1 \in M_G$\\
		Induktionsvorraussetzung:\\
		$G = 20\cdot a + 50\cdot b$\\
		Induktionsschluss:\\
		Fall 1: $G = 20\cdot (a + 1) + 50\cdot b = (20\cdot a + 50\cdot b) + (20\cdot 1 + 50\cdot 0) \in M_G$\\
		Fall 2: $G = 20\cdot a + 50\cdot (b + 1) = (20\cdot a + 50\cdot b) + (20\cdot 0 + 50\cdot 1) \in M_G$\\
		Fall 3: $G = 20\cdot (a + 1) + 50\cdot (b + 1) = (20\cdot a + 50\cdot b) + (20\cdot 1 + 50\cdot 1) \in M_G$\\[1em]
		Im Allgemeinen gilt also $(20\cdot a + 50\cdot b) + (20\cdot a_n + 50\cdot b_n) \in M_G,$ mit $a_n,b_n\in \mathbb{N}$
		Damit ist die Behauptung bewiesen.\\[1em]
	\end{flushleft}
	\begin{flushleft}
		\textbf{Aufgabe 3:}\\	
		zu zeigen: $21|(4^{n+1} + 5^{2\cdot n-1})$\\[1em]
		Induktionsanfang:\\
		$n=1\Rightarrow 21|(4^{1+1} + 5^{2\cdot 1-1}) = 21$\\[1em]
		Induktionvorraussetzung:\\
		$21|(4^{n+1} + 5^{2\cdot n-1}) =$ I.V.\\[1em]
		Induktionsschluss:\\
		$n+1:$ $4^{n+1+1} + 5^{2\cdot(n+1)-1}=$\\
		$=4^{n+2} + 5^{2\cdot n+1}=$\\
		$=4\cdot 4^{n+1} + 25\cdot 5^{2\cdot n-1}=$\\
		$=4\cdot 4^{n+1} + 4\cdot 5^{2\cdot n-1} + 21\cdot 5^{2\cdot n-1}=$\\
		$=4\cdot (4^{n+1} + 5^{2\cdot n-1}) + 21\cdot 5^{2\cdot n-1}=$\\
		$4\cdot (4^{n+1} + 5^{2\cdot n-1})=4\cdot$ I.V. (ist durch 21 teilbar) und $21| (21\cdot 5^{2\cdot n-1})$\\
		q.e.d.\\[1em]
	\end{flushleft}
	\begin{flushleft}
		\textbf{Aufgabe 4:}\\
		ges: alle $n$ der $f_n$ aller geraden ($2|f_n$) Fibonaccizahlen.\\
		Behauptung: Menge der $n$ aller gerade Fibonaccizahlen ist: $M_f= \{n:n=3\cdot i-1,i\in\mathbb{N}\}$\\[1em]
		Induktionsanfang:\\
		$i=1, n=3\cdot i-1=2$\\
		$f_{n-2} = 0 \Rightarrow 2\mid f_0 \Rightarrow n\in M_f$\\
		$f_{n-1} = 1 \Rightarrow 2\nmid f_1 \Rightarrow n\notin M_f$\\
		$f_{n} = f_1 + f_0 = 1 \Rightarrow 2\nmid f_2 \Rightarrow n\notin M_f$\\[1em]
		Induktionsvorraussetzung:\\
		$n= 3\cdot i-1,i\in\mathbb{N}$\\
		$f_{n} = f_{n-1} + f_{n-2}$\\[1em]
		Induktionsschluss:\\
		o.B.d.A fangen wir bei $i=0, n=2$ an. Das können wir machen, da 2 der kleine Index für eine gerade Fibonaccizahl ist. Für die nachfolgenden dreier Zahlenfolgen gilt dann...\\
		Fall 1: $f_{n+1} = f_{n} + f_{n-1}\Rightarrow 2\mid f_{n}\wedge 2\nmid f_{n-1}\Rightarrow 2\nmid f_{n+1}$\\
		Fall 2: $f_{n+2} = f_{n+1} + f_{n}\Rightarrow 2\nmid f_{n+1}\wedge 2\mid f_{n}\Rightarrow 2\nmid f_{n+2}$\\
		Fall 3: $f_{n+3} = f_{n+2} + f_{n+1}\Rightarrow 2\mid f_{n+2}\wedge 2\mid f_{n+1}\Rightarrow 2\mid f_{n+3}$\\[1em]
		$n+3=(i+1)\cdot 3-1$\\
		Somit ist gezeigt, dass jede dritte Fibonaccizahl ab der 2, durch 3 teilbar ist.\\[1em]
	\end{flushleft}
	\begin{flushleft}
		\textbf{Aufgabe 5:}\\	
		\begin{tabular}{|c|c|c|c|c|c|}
			 & Ebene & Kreis 1 & Kreis 2 & Kreis 3 & Kreis 4  \\
			neue Part.& 1 & 1 & 2 & 4 & 6\\
			insg. Part & 1 & 2 & 4 & 8 & 14  \\
		\end{tabular}\\[1em]
		Durch einen neuen Kreis entstehen neue Partition in den anderen Kreisen und in der Ebene.\\
		- $n-1$ Partitionen entstehen durch den Schnitt mit allen anderen Kreisen.\\
		- $n-1-1$ Partitionen entstehen durch den Schnitt mit den Schnitten des voran gegangenen Kreises.\\
		- $+1$ der Kreis.\\[1em]
		Anzahl der neuen Partitionen: $p = n + 1 + n - 1 - 1 + 1=2\cdot n - 2 + 1 = 2\cdot (n -1)$\\
		Anzahl aller Partitionen: $P=1+1+\sum\limits_{2}^{n} 2\cdot (i - 1)$\\
		Induktionsanfang:\\
		$n=2, 1+1+\sum\limits_{2}^{2} 2\cdot (2 - 1) = 4 = 2^2-2+2$\\
		Induktionsvorraussetzung:\\
		$p_n= 1+1+\sum\limits_{2}^{2} 2\cdot (i - 1), \forall n\in \mathbb{N},n\geq 2$\\
		Induktionsschluss:\\
		$p_{n+1}= 1+1+\sum\limits_{2}^{n+1} 2\cdot (i - 1)=1+1+\sum\limits_{2}^{n} 2\cdot (i - 1) + 2\cdot (n + 1 - 1)=n^2-n+2+2\cdot n=n^2+n+2=$\\
		$= n^2 + 2\cdot n -n + 1 - 2 + 2 = (n+1)^2 - (n+1)+2$\\[1em]
	\end{flushleft}
	\begin{flushleft}
		\textbf{Aufgabe 6:}\\	
		zu zeigen: $1 + \frac{1}{\sqrt{2}} + \frac{1}{\sqrt{3}} + \dots + \frac{1}{\sqrt{n}}> 2\cdot(\sqrt{n+1} -1)$\\[1em]
		Induktionsanfang:\\
		$n=0\Rightarrow 1 > 2\cdot(\sqrt{1} -1) = 0$\\[1em]
		Induktionsvorraussetzung:\\
		$1 + \frac{1}{\sqrt{2}} + \frac{1}{\sqrt{3}} + \dots + \frac{1}{\sqrt{n}}= \sum\limits_{k=0}^{n}\frac{1}{\sqrt{k}}> 2\cdot(\sqrt{n+1} -1)$\\[1em]
		Induktionsschritt:\\
		$\sum\limits_{k=0}^{n+1}\frac{1}{\sqrt{k}}> 2\cdot(\sqrt{n+2} -1)$\\
		$\equiv \sum\limits_{k=0}^{n}\frac{1}{\sqrt{k}} + \frac{1}{\sqrt{n+1}}> 2\cdot(\sqrt{n+2} -1)$\\
		$\equiv \sum\limits_{k=0}^{n}\frac{1}{\sqrt{k}} + \frac{1}{\sqrt{n+1}}> 2\cdot(\sqrt{n+1} -1) + \frac{1}{\sqrt{n+2}} > 2\cdot(\sqrt{n+2} -1)$\\
		$\equiv 2\cdot(\sqrt{n+1} -1) + \frac{1}{\sqrt{n+1}} > 2\cdot(\sqrt{n+2} -1)$\\
		$\equiv \frac{1}{\sqrt{n+1}} > 2\cdot(\sqrt{n+2} -1) - 2\cdot(\sqrt{n+1} -1)$\\
		$\equiv \frac{1}{\sqrt{n+1}} > 2\cdot(\sqrt{n+2}-\sqrt{n+1})$             ; wir wenden an $\rightarrow \cdot (\sqrt{n+2}+\sqrt{n+1})$\\
		$\equiv \frac{\sqrt{n+2}+\sqrt{n+1}}{\sqrt{n+1}} > 2\cdot(\sqrt{n+2}-\sqrt{n+1})\cdot(\sqrt{n+2}+\sqrt{n+1})$\\ 
		$\equiv \frac{\sqrt{n+2}}{\sqrt{n+1}}+1 > 2$\\ 
		Da $\sqrt{n+2}>\sqrt{n+1}\geq 1$ ist auch die Summe $>2$ und die Behauptung bewiesen.\\[1em]
	\end{flushleft}
\end{document}
