\documentclass[a4paper]{scrartcl}
\usepackage[utf8]{inputenc}
\usepackage[ngerman]{babel}
\usepackage[T1]{fontenc}
\usepackage{mathtools}
\usepackage{amsmath}
\usepackage{amssymb}
\usepackage{amsfonts}
\usepackage{polynom}

\title{Logik und diskrete Mathematik, Übung 7}
\author{Andreas Timmermann, Mat: 4994606, Alena Dudarenok, Mat: 4999780, Gruppe: 1}

\begin{document}
	\maketitle
	\begin{flushleft}
		\textbf{Aufgabe 1}\\
		\textbf{a)}\\
		
		\textbf{b)}\\
		zu zeigen: $(1-\sqrt{5})^n + (1+\sqrt{5})^n$ ist ganzzahlig.\\
		Beweis:\\
		$(1-\sqrt{5})^n + (1+\sqrt{5})^n = \sum\limits_{k=0}^{n}\binom{n}{k}1^{n-k}\cdot ((-1)\cdot \sqrt{5})^k + \sum\limits_{k=0}^{n}\binom{n}{k}1^{n-k}\cdot (\sqrt{5})^k$\\
		$= \sum\limits_{k=0}^{n}(\binom{n}{k}1^{n-k}\cdot ((-1)\cdot \sqrt{5})^k + \binom{n}{k}1^{n-k}\cdot (\sqrt{5})^k)$\\
		$= \sum\limits_{k=0}^{n}\binom{n}{k}(1^{n-k}\cdot ((-1)\cdot \sqrt{5})^k + 1^{n-k}\cdot (\sqrt{5})^k)$\\
		$= \sum\limits_{k=0}^{n}\binom{n}{k}(1^{n-k}\cdot (((-1)\cdot \sqrt{5})^k + (\sqrt{5})^k))$\\
		$= \sum\limits_{k=0}^{n}\binom{n}{k}(1^{n-k}\cdot (\sqrt{5})^k \cdot((-1)^k  + 1))$\\
		$\Rightarrow \forall k $ mit $2|k$ gilt $\binom{n}{k}(1\cdot (5^{\frac{k}{2}} \cdot(1  + 1)))$ ist ganzzahlig und $\forall k $ mit $2\nmid k$ gilt $\binom{n}{k}(1\cdot (\sqrt{5}^k \cdot(-1  + 1))) = 0$ ist auch ganzzahlig.\\
		$\Rightarrow \sum\limits_{k=0}^{n}\binom{n}{k}(1^{n-k}\cdot (\sqrt{5})^k \cdot((-1)^k  + 1))$\ ist ganzzahlig.\\
	\end{flushleft}
	\begin{flushleft}
		\textbf{Aufgabe 2}\\
	\end{flushleft}
	\begin{flushleft}
		\textbf{Aufgabe 3}\\
	\end{flushleft}
	\begin{flushleft}
		\textbf{Aufgabe 4}\\
	\end{flushleft}
	\begin{flushleft}
		\textbf{Aufgabe 5}\\
	\end{flushleft}
\end{document}
