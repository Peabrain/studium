\documentclass[a4paper]{scrartcl}
\usepackage[utf8]{inputenc}
\usepackage[ngerman]{babel}
\usepackage[T1]{fontenc}
\usepackage{mathtools}
\usepackage{amsmath}
\usepackage{amssymb}
\usepackage{amsfonts}
\usepackage{polynom}
\usepackage{stmaryrd}

\title{Logik und diskrete Mathematik, Übung 9}
\author{Andreas Timmermann, Mat: 4994606, Alena Dudarenok, Mat: 4999780, Gruppe: 3}

\begin{document}
	\maketitle
	\begin{flushleft}
		\textbf{Aufgabe 1}\\
		\textbf{a)}\\
		$x_{n+1} = \neg x_{n} \Leftrightarrow x_{n+1}=\begin{cases} 0, &\text{falls } x_{n} = 1 \\ 1, &\text{falls } x_{n} = 0 \end{cases}$\\[1em]
		Da immer einer der beiden Terme $x_n$ oder $x_{n+1}$ $true$ sein muss, wähle ich eine alternierende Methode, da auch die Ergebnisse der einzelnen Terme aus alternierenden Termen berechnet werden.\\
		\textbf{b)}\\
		(1) $(\neg q\vee r)\wedge \neg(q \vee p)\wedge \neg q = (\neg q\vee r)\wedge \neg q \wedge \neg p = (\neg q \wedge \neg p) \vee (r \wedge \neg q \wedge \neg p) = \neg q \wedge \neg p$\\
		(2) $[(p\vee q)\wedge(p\Rightarrow r)\wedge(q\Rightarrow r)]\Rightarrow r =$\\
		$= \neg[(p\vee q)\wedge(\neg p\vee r)\wedge(\neg q\vee r)]\vee r = $\\
		$= [\neg(p\vee q)\vee \neg(\neg p\vee r)\vee \neg(\neg q\vee r)]\vee r=$\\
		$= [(\neg p\wedge \neg q)\vee (p\wedge \neg r)\vee (q\wedge \neg r)]\vee r =$\\
		$= (\neg p\wedge \neg q) \vee r \vee (p \vee q)\wedge \neg r = 1$\\[1em]

	\end{flushleft}
	\begin{flushleft}
		\textbf{Aufgabe 2}\\
		Beweisen Sie: R ist reflexiv und zirkulär genau dann, wenn R Äquivalenzrelation ist.\\[1em]
		"$\Rightarrow$"\\
		zu zeigen ist nur die Symmetrie: $aRb \wedge bRb \Rightarrow bRa$\\[1em]
		"$\Leftarrow$"\\
		zu zeigen ist nur die Zirkularität: $aRb \wedge bRc \Rightarrow \underbrace{aRc \Rightarrow cRa}_{Symmetrie}$\\[1em]
		
	\end{flushleft}
	\begin{flushleft}
		\textbf{Aufgabe 3}\\
		\textbf{a)} \\
		\underline{reflexiv und symmetrisch aber nicht transitiv:}\\
		Relation: $R =$ Mensch $a$ ist befreundet mit Mensch $b$.\\
		\textbf{reflexiv:} $aRa$. Jeder Mensch ist befreundet mit sich selbst.\\
		\textbf{symmetrisch:} $aRb \Rightarrow bRa$. Wenn Mensch $a$ befreundet ist mit Mensch $b$, dann ist auch Mensch $b$ mit Mensch $a$ befreundet.\\
		\textbf{transitiv:} $aRb \wedge bRc\Rightarrow aRc$ $\mathbb{\lightning}$. Wenn Mensch $a$ mit Mensch $b$ befreundet ist und Mensch $b$ mit Mensch $c$ befreundet ist, muss Mensch $a$ nicht mit Mensch $c$ befreundet sein.\\[1em]
		\underline{nicht reflexiv aber symmetrisch und transitiv:}\\
	\end{flushleft}
	\begin{flushleft}
		\textbf{Aufgabe 4}\\
		\textbf{a)}\\
		geg: $H_k=1+\frac{1}{2}+\frac{1}{3}+\cdots +\frac{1}{k}, k\ge 1, k$-te harmonische Zahl\\
		zu zeigen: $n\ge 0, H_{2^n}\le 1+n$\\[1em]
		\underline{Induktionsanfang:} $n=0, H_{2^0} = H_{1} = 1\le 1 + 0$\\ 
		\underline{Induktionsvoraussetzung:} $H_{2^i}\le 1+i, 0\le i \le n$\\
		\underline{Induktionsschritt:} $H_{2^{n+1}}= \underbrace{1+\frac{1}{2}+\frac{1}{3}+\cdots +\frac{1}{2^n}}_{Induktionsvoraussetzung}+\frac{1}{2^n+1}+\cdots +\frac{1}{2^{n+1}}$\\
		$\le 1+n+\underbrace{\frac{1}{2^n+1}+\cdots +\frac{1}{2^{n+1}}}_{2^n-viele}$\\
		$\le 1+n+2^n\cdot \frac{1}{2^n+1} = 1+n+\frac{2^n}{2^n+1}$\\
		$\le 1+(n+1)$\\
		\begin{center}
		\textbf{q.e.d.}\\
		\end{center}
		\textbf{b)}\\
		zu zeigen: $1+\frac{1}{2^2}+\frac{1}{3^2}+\cdots +\frac{1}{n^2} < 2-\frac{1}{n}, n>1$\\
		\underline{Induktionsanfang:} $1+\frac{1}{2^2} = 1.25 < 2 - \frac{1}{2} = 1.5$\\
		\underline{Induktionsvoraussetzung:} $1+\frac{1}{2^2}+\frac{1}{3^2}+\cdots +\frac{1}{2^i} < 2-\frac{1}{i}, 1< i \le n$\\
		\underline{Induktionsschritt:} $n+1 \Rightarrow \underbrace{1+\frac{1}{4}+\frac{1}{9}+\cdots +\frac{1}{n^2}}_{Induktionsvorraussetzung}+\frac{1}{(n+1)^2}< 2-\frac{1}{n}+\frac{1}{(n+1)^2}$\\[1em]
		Jetzt zeigen wir noch: $2-\frac{1}{n}+\frac{1}{(n+1)^2} < 2 - \frac{1}{n+1}$\\
		\begin{center}
		\begin{tabular}{rcl | l}
		$2-\frac{1}{n}+\frac{1}{(n+1)^2}$ & $<$ & $2 - \frac{1}{n+1}$ & $-2$\\
		$-\frac{1}{n}+\frac{1}{(n+1)^2}$ & $<$ & $-\frac{1}{n+1}$ & $+\frac{1}{n}, +\frac{1}{n+1}$\\
		$\frac{1}{(n+1)^2}+\frac{1}{n+1}$ & $<$ & $\frac{1}{n}$ & $\frac{1}{n+1} \cdot \frac{n+1}{n+1}$\\
		$\frac{1}{(n+1)^2}+\frac{n+1}{(n+1)^2}$ & $<$ & $\frac{1}{n}$ & \\
		$\frac{n+2}{(n+1)^2}$ & $<$ & $\frac{1}{n}$ & $\cdot n$\\
		$\frac{n\cdot(n+2)}{(n+1)^2}$ & $<$ & $1$ & \\
		$\frac{n^2+2\cdot n}{(n+1)^2}$ & $<$ & $1$ & \\
		$\frac{n^2+2\cdot n+1-1}{(n+1)^2}$ & $<$ & $1$ & \\
		$\frac{(n+1)^2-1}{(n+1)^2}$ & $<$ & $1$ & \\
		$\frac{(n+1)^2-1}{(n+1)^2}$ & $<$ & $1$ & \\
		$\frac{(n+1)^2}{(n+1)^2}-\frac{1}{(n+1)^2}$ & $<$ & $1$ & \\
		$1-\frac{1}{(n+1)^2}$ & $<$ & $1$ & \\
		\end{tabular}\\[1em]
		\textbf{q.e.d.}\\
		\end{center}
	\end{flushleft}
	\begin{flushleft}
		\textbf{Aufgabe 5}\\
	\end{flushleft}
\end{document}
